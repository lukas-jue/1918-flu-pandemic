\documentclass[12pt,a4paper]{article}
\usepackage{tabularx}
\usepackage{booktabs}
\usepackage{longtable}
\usepackage{ltxtable}
\usepackage[latin1]{inputenc}
\usepackage{amssymb}
\usepackage[]{graphicx,rotating}
%\usepackage[T1]{fontenc}
\usepackage{parskip}
\usepackage{natbib}
\usepackage[official]{eurosym}
%\usepackage{mathrsfs}
%\usepackage{amsmath}
\usepackage{verbatim}
\usepackage{epstopdf} % to include eps files
\usepackage{enumitem} % for the abbreviations table
\usepackage{helvet} %Arial
\renewcommand{\familydefault}{\sfdefault} %Arial
%\usepackage{mathptmx} %Times New Roman
%\renewcommand{\familydefault}{cmr} % Computer Modern. alternative: cmss

\usepackage[left=2.5cm, right=2.5cm, top=2.5cm]{geometry}

\pagestyle{empty}
\parindent 0.0cm
\renewcommand{\baselinestretch}{1}
\newcommand{\bs}{\boldsymbol}
\pdfminorversion=7
\bibliographystyle{agsm}

%nice quotes
\let\oldquote\quote
\let\endoldquote\endquote
\renewenvironment{quote}[2][]
{\if\relax\detokenize{#1}\relax
	\def\quoteauthor{#2}%
	\else
	\def\quoteauthor{#2~---~#1}%
	\fi
	\oldquote}
{\par\nobreak\smallskip\hfill(\quoteauthor)%
	\endoldquote\addvspace{\bigskipamount}}

\begin{document}

\begin{center}
% \vspace*{1cm}
 \includegraphics[width=0.2\textwidth]{external/GU-Logo-blau-CMYK.eps} \vspace{1.5cm}
  
{\large{\bf Economic Consequences of the 1918 Flu Pandemic:\\
		History's Lesson for the Current Crisis}}

\textbf{Lukas J\"urgensmeier} \\
{\footnotesize (Mat.-Nr.: 6904281)}

  Term Paper \\\vspace{1.5cm}
  
  \begin{abstract}
  	The ongoing coverage of Covid-19 often describes the current crisis as \textit{unprecedented}.
  	However, there exists precedent: The 1918 flu pandemic.
  	This paper first examines the economic consequences of the pandemic one century ago.
  	Second, it highlights to which extent lessons from the 1918 flu pandemic could be transferred to efficiently handle the current crisis.
  \end{abstract} \vspace{1.5cm}
  
  submitted to \\\vspace{0.5cm}
  \textbf{Prof. Dr. Volker Casperi} \\
  Graduate School of Economics, Finance, and Management \\
  Goethe University \\
  Frankfurt am Main \vspace{1.5cm}
  
  in fulfillment of the requirements \\
  of the lecture \\\vspace{0.5cm}
  \textbf{Historical and Normative Foundations of Economics} \\\vspace{0.5cm}
  July 30, 2020
  
\end{center}


\pagebreak
\pagestyle{plain}
\pagenumbering{Roman}
\tableofcontents
\pagebreak
\listoffigures
\listoftables
\newpage
\newlist{abbrv}{itemize}{1}
\setlist[abbrv,1]{label=,labelwidth=1in,align=parleft,itemsep=0.1\baselineskip,leftmargin=!}
 
\section*{List of Abbreviations}
%\chaptermark{List of Abbreviations}
 
\begin{abbrv}
 
\item[ABC]			Placeholder

 
\end{abbrv}
\newpage
\setcounter{page}{2}
\pagenumbering{arabic}
\setlength{\baselineskip}{1.5\baselineskip}
\pagestyle{plain}


\section{Introduction}

\begin{quote}{Mark Twain, 1910}
	\textit{"History doesn't repeat itself, but it rhymes."}
\end{quote}


\section{Development of the 1918 Flu Pandemic}

\section{Economic Consequences of the 1918 Flu Pandemic}
\subsection{Short-term: During the Outbreak in 1918-19}
\subsubsection{Health Care Costs}
\subsubsection{Non-Pharmaceutical Interventions}

\begin{itemize}
	\item \cite{correiaPandemicsDepressEconomy2020}:
	Using geographic variation in mortality during the 1918 Flu Pandemic in
	the U.S., we find that more exposed areas experience a sharp and persistent decline
	in economic activity. The estimates imply that the pandemic reduced manufacturing
	output by 18 percent. The downturn is driven by both supply and demand-side channels.
	Further, building on findings from the epidemiology literature establishing that NPIs
	decrease influenza mortality, we use variation in the timing and intensity of NPIs
	across U.S. cities to study their economic effects. We find that cities that intervened
	earlier and more aggressively do not perform worse and, if anything, grow faster after
	the pandemic is over. Our findings thus indicate that NPIs not only lower mortality;
	they also mitigate the adverse economic consequences of a pandemic
	
	Altogether, our	findings suggest that pandemics can have substantial economic costs, and NPIs can have
	economic merits, beyond lowering mortality.
\end{itemize}

\subsection{Medium- to Long-term: After the Immediate Health Thread Subsided}

\subsubsection{Negative Impact on Fetal Health}
\begin{itemize}
	\item \cite{almond1918InfluenzaPandemic2006}: cohorts in utero during
	the pandemic displayed reduced educational attainment, increased
	rates of physical disability, lower income, lower socioeconomic status,
	and higher transfer payments compared with other birth cohorts.
	
	likelihood of being poor rose as much as 15 percent
	
	Prenatal exposure to the 1918 influenza pandemic had large negative
	effects on adult economic outcomes. 
\end{itemize}

\subsubsection{Loss of Life}

\begin{itemize}
	\item \cite{fanInclusiveCostPandemic2016}: In this paper we use findings from that literature to generate an
	estimate of pandemic cost that is inclusive of both income loss and the cost of elevated mortality.
	
	For moderately severe pandemics about 40\% of inclusive cost results from income loss. For
	severe pandemics this fraction declines to 12\%: the intrinsic cost of elevated mortality becomes
	completely dominant.
\end{itemize}

\subsubsection{Interest Rates, Wages, and Asset Prices}

\begin{itemize}
	\item \cite{brainerdEconomicEffects19182003}: large and robust positive effect of the influenza epidemic on per capita income growth across states during the 1920s. 
	\item \cite{karlssonImpact1918Spanish2014}: We find that the pandemic led to a significant increase in poverty rates.
	There is also relatively strong evidence that capital returns were negatively affected by the	pandemic. However, we find robust evidence that the influenza had no discernible effect on earnings. This finding is surprising since it goes against most previous empirical studies as well as theoretical predictions.
	
	\item \cite{jordaLongerrunEconomicConsequences2020}: We study rates of return on assets
	using a dataset stretching back to the 14th century, focusing on 15 major pandemics
	where more than 100,000 people died. In addition, we include major armed conflicts
	resulting in a similarly large death toll. Significant macroeconomic after-effects of the
	pandemics persist for about 40 years, with real rates of return substantially depressed.
	In contrast, we find that wars have no such effect, indeed the opposite. This is consistent
	with the destruction of capital that happens in wars, but not in pandemics. Using
	more sparse data, we find real wages somewhat elevated following pandemics. The
	findings are consistent with pandemics inducing labor scarcity and/or a shift to greater
	precautionary savings.
\end{itemize}

\section{Parallels and Differences of the 1918 Flu Pandemic and Covid-19}
\begin{itemize}
	\item Insurance modelers developed a framework to simulate an outbreak of the 1918 flu pandemic today \citep{madhavModelingModernDay2013}. Differences in population groups affected. 1918 mostly young adults and few older, 2020 mostly older adults.	Implications for different economic consequences: Today, less adults in the phase of prime economic productivity seem die from the disease.
	\item Timing right after World War II: Economy was already in a recession, which masked the economic shock due to the pandemic.
	Contrary, Covid-19 hit the world economy during a decade-long expansion.
	Hence, the economic stakes were higher and the immediate impact on the economy more visible.
\end{itemize}






\section{Conclusion}

\clearpage
\appendix
\section{Appendix}

\clearpage
\bibliography{library}

\newpage
\thispagestyle{empty}
\section*{Statutory Declaration}

Ich versichere hiermit, dass ich die vorliegende Arbeit selbs\"andig und ohne Benutzung anderer als der angegebenen Quellen und Hilfsmittel verfasst habe. W\"ortlich \"ubernommene S\"atze oder Satzteile sind als Zitat belegt, andere Anlehnungen, hinsichtlich Aussage und Umfang, unter Quellenangabe kenntlich gemacht. Die Arbeit hat in gleicher oder \"ahnlicher Form noch keiner Pr\"ufungsbeh\"orde vorgelegen und ist nicht ver\"offentlicht. Sie wurde nicht, auch nicht auszugsweise, f\"ur eine andere Pr\"ufungs- oder Studienleistung verwendet.


I herewith declare that I have completed the present term paper independently, without making use of
other than the specified literature and aids. Sentences or parts of sentences quoted literally are
marked as quotations; identification of other references with regard to the statement and scope of
the work is quoted. The thesis in this form or in any other form has not been submitted to an examination body and has not been published.
This thesis has not been used, either in whole or part, for another examination achievement.

\vspace{1cm}

Frankfurt am Main, July 30, 2019

\includegraphics[scale=0.12]{external/signature.png}\\
Lukas J\"urgensmeier
\end{document}
