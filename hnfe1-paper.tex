\documentclass[12pt,a4paper]{article}
\usepackage{tabularx}
\usepackage{booktabs}
\usepackage{longtable}
\usepackage{ltxtable}
\usepackage[latin1]{inputenc}
\usepackage{amssymb}
\usepackage[]{graphicx,rotating}
%\usepackage[T1]{fontenc}
\usepackage{parskip}
\usepackage{natbib}
\usepackage[official]{eurosym}
%\usepackage{mathrsfs}
%\usepackage{amsmath}
\usepackage{verbatim}
\usepackage{epstopdf} % to include eps files
\usepackage{enumitem} % for the abbreviations table
%\usepackage{helvet} %Arial
%\renewcommand{\familydefault}{\sfdefault} %Arial
%\usepackage{mathptmx} %Times New Roman
%\renewcommand{\familydefault}{cmr} % Computer Modern. alternative: cmss

\usepackage[left=2.5cm, right=2.5cm, top=2.5cm]{geometry}

\pagestyle{empty}
\parindent 0.0cm
\renewcommand{\baselinestretch}{1}
\newcommand{\bs}{\boldsymbol}
\pdfminorversion=7
\bibliographystyle{agsm}

%nice quotes
\let\oldquote\quote
\let\endoldquote\endquote
\renewenvironment{quote}[2][]
{\if\relax\detokenize{#1}\relax
	\def\quoteauthor{#2}%
	\else
	\def\quoteauthor{#2~---~#1}%
	\fi
	\oldquote}
{\par\nobreak\smallskip\hfill(\quoteauthor)%
	\endoldquote\addvspace{\bigskipamount}}

\begin{document}

\begin{center}
% \vspace*{1cm}
 \includegraphics[width=0.2\textwidth]{external/GU-Logo-blau-CMYK.eps} \vspace{1.5cm}
  
{\large{\bf Economic Consequences of the 1918 Flu Pandemic:\\
		Reviewing Evidence from the Historical Precedent of Covid-19}}

\textbf{Lukas J\"urgensmeier} \\
{\footnotesize (Mat.-Nr.: 6904281)}

  Term Paper \\\vspace{1.5cm}
  
  \begin{abstract}
  	The ongoing coverage of Covid-19 often describes the current pandemic and its economic consequences as \textit{unprecedented}.
  	However, there exists precedent: The 1918 flu pandemic.
  	By examining the existing literature on its economic effects, this paper highlights a consensus that most indicators were worsening (GDP, consumption, poverty rates, capital returns); possibly except for increased wages due to a labor supply shock.
  	However, estimates vary largely by study, evidence remains scant and inconclusive, so generalizations might be difficult.
  	To that end, this paper secondly discusses similarities and differences between the 1918 flu pandemic and Covid-19.
  	I conclude that while the pandemics might be similar at the first look, a direct projection from 1918 to 2020 is difficult and economic predictions based on the historical event might be imprecise.
  	The main reasons are an unusually high mortality rate in 1918--19 of prime working age individuals, combined with the vastly different economic, medical, and societal conditions at the end of World War I compared to today.
  	
  \end{abstract} \vspace{1.5cm}
  
  submitted to \\\vspace{0.5cm}
  \textbf{Prof. Dr. Volker Casperi} \\
  Graduate School of Economics, Finance, and Management \\
  Goethe University \\
  Frankfurt am Main \vspace{1.5cm}
  
  in fulfillment of the requirements \\
  of the lecture \\\vspace{0.5cm}
  \textbf{Historical and Normative Foundations of Economics} \\\vspace{0.5cm}
  July 30, 2020
  
\end{center}


\pagebreak
\pagestyle{plain}
\pagenumbering{Roman}
\tableofcontents
\pagebreak
\listoffigures
\listoftables
\newpage
\newlist{abbrv}{itemize}{1}
\setlist[abbrv,1]{label=,labelwidth=1in,align=parleft,itemsep=0.1\baselineskip,leftmargin=!}
 
\section*{List of Abbreviations}
%\chaptermark{List of Abbreviations}
 
\begin{abbrv}
 
\item[ABC]			Placeholder

 
\end{abbrv}
\newpage
\setcounter{page}{2}
\pagenumbering{arabic}
\setlength{\baselineskip}{1.5\baselineskip}
\pagestyle{plain}


\section{Introduction}

\begin{quote}{Mark Twain, 1910}
	\textit{"History doesn't repeat itself, but it rhymes."}
\end{quote}

The late 1910s were marked by devastation.
The first truly global conflict, World War I, exerted a huge human toll between 1914 and 1918:
Historians estimate approximately 40 million civilian and military causalities resulting from the war \citep{royde-smithWorldWarKilled2020}.
However, a significant share of those deaths is not directly attributed to military conflict,
but to the second devastating event overlapping with the end of the military conflict and lasting at least until 1919: The 1918 flu pandemic.

\section{Development of the 1918 Flu Pandemic}

Beginning in March 1918, a novel influenza virus likely emerged in mid-western parts of the United States \citep{barrySiteOrigin19182004} and spread virulently, possibly sped up by large troop movements to Europe due to World War I. It reached almost every part of the world within six months \citep{pattersonGeographyMortality19181991}.
In mainly three waves in spring 1918, fall 1918, and winter/spring 1918/19 in the northern hemisphere, the virus infected approximately 500 million people, roughly one third of the entire world population at that time \citep{taubenberger1918InfluenzaMother2006}.
Earlier research estimated the global death toll from the virus between 21.5 \cite{jordanEpidemicInfluenzaSurvey1927} and later up to 39.3 million \citep{pattersonGeographyMortality19181991}.
More recent studies, while acknowledging the large uncertainty, account for greater under-reporting of causalities and hence estimate the global death toll between 50 and 100 million \citep{johnsonUpdatingAccountsGlobal2002}.
It is therefore reasonable to assume that the health emergency caused more causalities that the enormous toll of World War I.

While influenza outbreaks before and after the 1918 flu pandemic mainly caused worse courses of the disease and a larger death toll on the very young and older population ("U-shape"),
the outbreak in 1918 displayed an uncommon W-shape by additionally killing many otherwise healthy young adults aged 20-40 \citep{taubenberger1918InfluenzaMother2006}\footnote{
	see figures 2 and 3 in \cite{taubenberger1918InfluenzaMother2006} for detailed age profiles of incidences and causalities of the 1918 flu pandemic in comparison to the normal influenza shortly before and after.}.

Especially this ruthlessness on people in their prime working age came with significant implications for the economy.
While this section focused on the broader health consequences of the pandemic, the next part examines the economic consequences of the devastating 1918 flu pandemic.

\section{Economic Consequences of the 1918 Flu Pandemic}

By infecting one third of the world population and killing up to 100 million people, the 1918 flu pandemic not only caused grave health-related suffering, but also significantly influenced the course of the world economy.
This section first examines the economic short-term consequences arising during the pandemic, highlighting increased health care costs, higher rates of absenteeism, and the immediate negative economic effects of non-pharmaceutical interventions (NPI) such as quarantines and temporary business shutdowns.
In subsection \ref{sec:medlong}, attention turns to the medium- to long-term economic effects including the development of economic indicators such as interest rates, wages, and asset prices. 
Further, it discusses loss of life and a potential negative impact on fetal health as factors adversely affecting the long-term economic course.


	\subsection{Short-term: During the Outbreak in 1918--19} \label{subsec:short}

In absence of medical tools to end the outbreak such as vaccines or an effective treatment, one of the only options  immediately available to slow the spread are non-pharmaceutical interventions, including quarantines and mandatory social distancing rules \citep{aledortNonpharmaceuticalPublicHealth2007}.
Isolating the effect of NPIs on health outcomes has been and continues to be a challenge in the medical literature:
As one of the first studies attempting to establish causality from the 1918 flu pandemic, \cite{markelNonpharmaceuticalInterventionsImplemented2007} claim that US cities with early and continuous NPI measures overall had better health outcomes.
In a direct reply, \cite{barryCommentsNonpharmaceuticalInterventions2007}, one of the leading scholars on the 1918 flu pandemic\footnote{
	\cite{barryGreatInfluenzaEpic2005} is considered to be one of the most extensive and complete historical accounts of the 1918 flu pandemic \citep[see comment by the editor]{barryCommentsNonpharmaceuticalInterventions2007}.}, questions the methodology of the study.
While he does not per se challenge the notion that NPIs might have had a positive effect on slowing the spread and decreasing the negative health outcomes of the pandemic, he concludes that their methodology does not fulfill scientific standards of establishing a causal relationship.
This debate exemplifies how difficult a causal statement about the usefulness of NPIs is in this case due to limited historical data\footnote{In his reply, \cite{barryCommentsNonpharmaceuticalInterventions2007} notes that there have only been three significant pandemics in the last century, making the task of establishing causality hard from a mere statistical perspective.}.

While medical literature struggles with assessing the efficacy of NPIs, Economists have used the natural experiment resulting from the exogenous pandemic shock to produce robust evidence of the 1918 flu pandemic's economic consequences in general, and NPIs in particular.
Center at this debate is a trade-off policy-makers face: NPIs result in closed businesses and less immediate economic activity, but also possibly in a less severe course of the disease, fewer deaths, and hence a quicker recovery.

Regardless of the public health aspect, no clear economic intuition is available about whether the direct economic costs outweigh the indirect economic benefits through an earlier end to the pandemic.
Addressing this question, \cite{correiaPandemicsDepressEconomy2020} used the different timing of NPI adoption in US cities in 1918--19 to isolate the economic consequences of those measures.
They show that cities with an aggressive and timely intervention in form of NPIs not only had a lower mortality rate in 1918--19, but also did not perform worse in the short-term compared to late-adopters\footnote{\cite{correiaPandemicsDepressEconomy2020} furthermore suggest that long-term economic outcomes were significantly better in cities acting early, which is further discussed in section \ref{sec:medlong}.}.

In general, pandemics are economically characterized as a simultaneous shock to both supply and demand \citep{eichenbaumMacroeconomicsEpidemics2020}.
Workers will reduce their labor supply either due to risk of infection or by actually contracting the disease.
Simultaneously, demand decreases since consumers cannot, e.g. due to quarantines or social distancing measures, or do not want to consume goods and services, e.g. due to perceived risk of contracting the virus. While this can cause a persistent, long-term recession \citep{eichenbaumMacroeconomicsEpidemics2020}, especially the short-term labor supply shock was highly visible during the 1918 flu pandemic.

Since the prime working age population's health was severely negatively affected through the pandemic \citep{taubenberger1918InfluenzaMother2006}, industries struggled with an acute shortage of labor supply.
Absentee rates in key industries were strikingly high: According to \cite{barryPandemicsAvoidingMistakes2009}, 
shipyard workers in the United States were explicitly told that their work was as essential as a soldiers' duties, physicians were on-site, and they only received pay if present.
Nevertheless, 45\% -- 58\% were absent at the peak of the pandemic in late 1918 \citep{turnerReportPreventiveMeasure1919}.
While this is inconclusive evidence of a general labor supply shock, it provides one of the few available data points pointing into this direction.

Due to lack of unambiguous data on the short-term effects, economists have highlighted individual anecdotal evidence through newspaper articles.
\cite{garrettEconomicEffects19182007} reviews newspapers in 1918--19 in Little Rock, Arkansas, and Memphis, Tennessee, mentioning adverse effects on railway, mining and telephone service through lack of labor supply.
Additionally, this source summarizes reported negative effects on consumption, noting that local department stores and merchants saw a decrease in revenue between 40\% and 70\%.

Reading those individual historical accounts, one could suspect broader implications for the overall economy.
However, \cite{benmelech1918InfluenzaDid2020} note that contrary to that expectation,
at least the US economy did not suffer to the expected degree.
The authors highlight World War I and the associated major role of the US government as the main procurer and significant employer that did not cut back spending due to the pandemic.
In 1918, 38\% of GDP were real government expenditures and the military directly employed 6\% of the American workforce \citep{benmelech1918InfluenzaDid2020}.
While individual war-unrelated industries and businesses were hit by the pandemic, rebound was quick and the aggregate economy even grew by approximately 1\% in 1919 \citep{romerWorldWarPostwar1988}.
However, this growth rate was not caused by, but rather occurred \textit{despite} of the pandemic.
Also, the United States might have been an outlier globally:
\cite{barroCoronavirusGreatInfluenza2020} look at 42 countries and find a negative average pandemic effect on GDP of 6\% and approximately 8\% less consumption, isolated from negative World War I effects.
However, those estimates were derived from a sample between 1901 and 1929, and thus would be better categorized as medium- to long-term consequences, on which the following section focuses.


\subsection{Medium- to Long-term: After the Immediate Health Thread Subsided} \label{sec:medlong}

While the short-term disruptions to the economy are described as having an overall negative impact, medium- and long-term consequences contrarily were both negative and positive in terms of the economy.
This section discusses literature that supports the notion of a negative 1918 flu pandemic-impact on indicators such as GDP, capital returns, consumption \citep{karlssonImpact1918Spanish2014, barroCoronavirusGreatInfluenza2020, correiaPandemicsDepressEconomy2020} and income loss \citep{fanInclusiveCostPandemic2016}.
However, literature also acknowledges a possible positive impact on wages through decreased labor supply \citep{brainerdEconomicEffects19182003}.
Those findings seem to be consistent for previous major pandemics \citep{jordaLongerrunEconomicConsequences2020}.
Lastly, this part concludes with hypothesized very long-term negative consequences of the 1918 flu pandemic through in utero exposure to the disease and related impact on socioeconomic outcomes \citep{almond1918InfluenzaPandemic2006}.

The previous section already hinted at the literature that suggests a prolonged negative impact of the 1918 flu pandemic on most economic indicators.
Despite its historic significance, literature has only started paying closer attention on the economic consequences of the pandemic in 1918--19 during the past two decades\footnote{\cite{brainerdEconomicEffects19182003} note that prior to 2002, only \textit{two} known economics papers dealt with the 1918 flu pandemic's effects on the economy and neither of the three leading economic history textbooks at that time even mentioned the event.} and even more so recently due to Covid-19.
\cite{karlssonImpact1918Spanish2014} look at well-documented economic performance in Sweden around the 1918 pandemic, exploiting both the fact that the country was neutral in World War I and the large geographic differences in severity and timing of the disease to isolate the pandemic's direct effect on economic indicators.
They estimate an 11\% reduction of capital returns in worse affected counties compared to lesser struck areas of Sweden\footnote{more precisely, this estimate refers to the comparison of two counties by excess mortality rate. There is an 11\% difference in capital returns between the 75\%- and the 25\%-quantile county ordered by excess mortality.}.
While this effect is mainly during the pandemic, they find evidence of a smaller permanent reduction in the medium-term.
Contrarily, they find a strong and persistent medium-term increase in poverty rates that only becomes visible after the 1918 flu pandemic.
Acknowledging one possible mechanism that leads to increased poverty---deceased breadwinners that leave dependents behind without income---\cite{karlssonImpact1918Spanish2014} argue that the effect was way larger than could have been explained by this mechanism and therefore attribute most of the increased poverty to the pandemic \textit{per se}.

However, this finding is at first sight contradictory to a 1918 flu pandemic-induced medium-term increase in wages.
The 1918 flu pandemic exercised a large death toll on working people in their 20s and 30s, additionally sickening a large amount temporarily \citep{taubenberger1918InfluenzaMother2006}. 
While the second mechanism impacted only the short-term economy\footnote{
	refer to section \ref{subsec:short} for the short-term consequences.},
the death of a significant share of working population led to a well-documented and lasting labor supply shock.
\cite{brainerdEconomicEffects19182003} find a large positive pandemic-effect on per capita income growth in the United States during the 1920s.
Though the effect size of the pandemic varies largely by empirical specification, this paper does not specify further which point-estimate is considered the most realistic one.
Contrary to those results, \cite{karlssonImpact1918Spanish2014} find no conclusive evidence of increased wages in Sweden following the spread of the disease.

Though not intuitive at first, increased poverty rates are not mutually exclusive with higher wages.
Those two findings might be reconciled by hypothesizing an increased disparity in income:
People participating in the workforce might have been better off, while other parts of the population---e.g. families without a breadwinner or otherwise cut-off from economic participation---might have driven the increased poverty rate despite higher average wages.
However, empirical evidence remains scant and only looking at the raw estimates by \cite{pikettyDistributionalNationalAccounts2018} exhibit ambiguous developments:
The top 1\% US earners' income as a percentage of total income increased in 1919 by 2\%-points compared to the previous year, while decreasing to the earlier level of approximately 18\% in the following two years.
More research is therefore warranted to isolate the causal effect of the pandemic on income distribution in combination with poverty levels.

Another research thread tries to generalize economic effects of previous pandemics.
In theory, this might lead to more robust estimates through the increase in sample size and by not only focusing on a single event with unique medical characteristics.
On the other hand, data quality and consistency might suffer from comparing events such as the Black Death in the 14\textsuperscript{th} century to the H1N1 pandemic in 2009.
Doing so, \cite{jordaLongerrunEconomicConsequences2020} review 14 major pandemics with more than 100,000 deaths since the 14\textsuperscript{th} century and find a modest increase in wages that stretches over decades after a pandemic ended and estimate that this effect peaks roughly 35--40 years after.
In the same analysis, they estimate depressed real interest rate responses to pandemics with a peak approximately 25 years later and a magnitude of around 1--2\%.
In an attempt to check for robustness, they compare the results including and excluding the 1918 flu pandemic and reach the same conclusions, suggesting that this event had no structurally different economic consequences than the other studied pandemics.

Focusing more generally on possible future pandemics, \cite{fanInclusiveCostPandemic2016} use the 1918 flu pandemic to estimate \textit{inclusive} costs of a pandemic, measuring not only decrease in GDP, but also accounting for the costs due to premature mortality.
They argue that only around 40\% of the total cost of moderately severe pandemics is the easily measurable income\footnote{\cite{fanInclusiveCostPandemic2016} use the term \textit{income} for Gross National Income (GNI) instead of GDP, not to be confused with wages.}
decrease, on which the sparse \textit{pandemic economics} literature typically focused on.
They argue that most models failed to capture the intrinsic economic value of life and therefore draw from environmental and medical economics literature that quantifies the statistical value of a shortened life.
Using those estimates, their approach models the cost due to premature deaths during a pandemic and compares this amount to the direct effects on GDP.
Overall, \cite{fanInclusiveCostPandemic2016} find that this often neglected cost of a pandemic in most cases dominates the directly measurable effects on GDP.
Furthermore, they estimate that the more severe a pandemic becomes, the larger the share of cost due to loss of life becomes: In their most severe scenario, only 12\% of pandemic-related costs arise due to national income loss.

Most of the literature focuses on the effects of the pandemic maximum 40 years after the event ended.
Taking a different approach, \cite{almond1918InfluenzaPandemic2006} uses the three US 1960--1980 censuses to estimate the very long-term consequences beyond 40 years of \textit{in utero} exposure to the 1918 flu virus.
By comparing birth cohorts close to the pandemic with the ones whose mothers could have been infected, the author finds consistently worse socioeconomic outcomes for the affected cohort in almost all indicators recorded.
Long-term, he concludes that the likelihood of being poor increase by up to 15\%, children affected are 15\% less likely to finish high school, and male wages are 5--9\% lower due to in utero exposure to the virus.
Notably, the 1918 flu cohort had a 20\% higher rate of disability in the 1980 census compared to the otherwise equal cohorts shortly before and after the pandemic.
While the other short- and mid-term effects of the pandemic appear mostly intuitive, this study sheds a light on possibly often omitted very long-term health and economic disadvantages that the affected cohorts carry over long after the pandemic ended.


\section{Differences of the 1918 Flu Pandemic to Covid-19}

It might be tempting to map the economic consequences of the 1918 flu pandemic to today's crisis.
After all, both are considered worldwide pandemics and are respiratory diseases caused by a virulent strain of the influenza.
This section outlines why this would ignore several key differences in the medical characteristics of both diseases and structurally different social conditions, that combined make a direct comparison of the economic impact more difficult.

One of the surprising medical characteristics of the 1918 virus remains the unusual deadliness for patients in their 20s and 30s \citep{taubenberger1918InfluenzaMother2006}.
Contrarily, preliminary results show that the Covid-19 mortality distribution has its highest mass on the higher end of the population age and the probability of death for people far away from retirement age remains low \citep{weissClinicalCourseMortality2020, zhouClinicalCourseRisk2020}.
This has far reaching implications for the economic development.
One effect of the 1918 flu pandemic was a medium- to long-term labor supply shock since a significant share of the working population deceased as a result of infection.
As discussed in the previous section, this might have led to substantially higher wages in the following years, making it one of the few economic silver lining from a workers' point of view.
Health-wise, today's younger generation is more lucky.
However, this might translate to even more devastating labor market consequences, since the labor supply shock might be small and, if existent at all, only temporary.

Nevertheless, other conditions accompanying the pandemic draw a more optimistic picture.
Most notably, the Covid-19 pandemic emerged during a decade-long economic expansion, while the 1918 flu pandemic coincided with the final year of World War I.
The conflict led officials in many participating countries to censor the press in an attempt of keeping up the public's morale \citep{chafeeFreedomSpeechWar1919}.
Researchers cite this fact as a possible reason for a worse course of the disease \citep{madhavModelingModernDay2013}, since the population was not prepared and aware of the virus early enough. This possibly in turn led to worse economic consequences than one would expect with a well-aware public.
Compared to the wartime year 1918, today's press is freer and through technological advances reports quicker on the developments, ensuring a better-informed public.


\section{Conclusion}

This paper has shown that there appears to be a consensus in the literature: The 1918 flu pandemic in particular and pandemics in general overall negatively affect the economy with the possible exception from workers' point-of-view by increased wage levels due to labor supply shortages.
However, this part of the economic literature appears to be at an early stage.
The scarceness of historical precedent combined with the lack of detailed economic data on pandemics long time ago makes it challenging to reliably estimate effects across pandemics or countries.
Furthermore, drawing conclusions for the current and future pandemics remains a difficult endeavor.
Many structural differences between the 1918 and 2020 pandemics make a sound economic prediction a challenging endeavor.
Medical virus characteristics differ, such as the different mortality of the virus for different segments of the population and medicine has advanced hugely in the previous 100 years.
Those non-economic differences might imply vastly different economic outcomes.
Also, technological advances outside medicine, notably the possibility of many white-collar jobs to continue working from home make a direct prediction of the economic consequences based on the historical data very difficult.
Though there already exist several promising approaches to derive sound policy recommendations for the current pandemic, bridging the gap between the two pandemics often requires educated guesswork.
Given the significant uncertainty that society faces today, an expedited further examination of the historical precedent of Covid-19 is warranted and might help to increase the quality of the educated policy guesses with sound scientific advice.


\clearpage
\appendix
\section{Appendix}

\clearpage
\bibliography{library}

\newpage
\thispagestyle{empty}
\section*{Statutory Declaration}

Ich versichere hiermit, dass ich die vorliegende Arbeit selbs\"andig und ohne Benutzung anderer als der angegebenen Quellen und Hilfsmittel verfasst habe. W\"ortlich \"ubernommene S\"atze oder Satzteile sind als Zitat belegt, andere Anlehnungen, hinsichtlich Aussage und Umfang, unter Quellenangabe kenntlich gemacht. Die Arbeit hat in gleicher oder \"ahnlicher Form noch keiner Pr\"ufungsbeh\"orde vorgelegen und ist nicht ver\"offentlicht. Sie wurde nicht, auch nicht auszugsweise, f\"ur eine andere Pr\"ufungs- oder Studienleistung verwendet.


I herewith declare that I have completed the present term paper independently, without making use of
other than the specified literature and aids. Sentences or parts of sentences quoted literally are
marked as quotations; identification of other references with regard to the statement and scope of
the work is quoted. The thesis in this form or in any other form has not been submitted to an examination body and has not been published.
This thesis has not been used, either in whole or part, for another examination achievement.

\vspace{1cm}

Frankfurt am Main, July 30, 2019

\includegraphics[scale=0.12]{external/signature.png}\\
Lukas J\"urgensmeier
\end{document}
